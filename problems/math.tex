
\documentclass[12pt,a4paper]{article}

\usepackage[T2A]{fontenc} % Поддержка русских букв
\usepackage[utf8]{inputenc} % Кодировка utf8
\usepackage[english, russian]{babel} % Языки: русский, английский
% \usepackage{pscyr} % Нормальные шрифты

\usepackage{amsmath, amssymb}
\usepackage{amscd}
\usepackage{amsthm}
\usepackage[left=1.5cm,right=2cm,top=1.5cm,bottom=1.5cm]{geometry}
\usepackage{geometry}
\usepackage{fancyhdr}
\usepackage{color}
\usepackage{graphicx}
\graphicspath{ {./pic/} }

\usepackage{xcolor}
\usepackage{hyperref}
% Цвета для гиперссылок
\definecolor{linkcolor}{HTML}{799B03} % цвет ссылок
\definecolor{urlcolor}{HTML}{799B03} % цвет гиперссылок
\hypersetup{pdfstartview=FitH,  linkcolor=linkcolor,urlcolor=urlcolor, colorlinks=true}

\def\Z{\mathbb{Z}}
\def\N{\mathbb{N}}
\def\R{\mathbb{R}}
\def\F{\mathbb{F}}
\def\Q{\mathbb{Q}}

\def\E{\mathrm{\mathbb{E}}}
\def\D{\mathrm{\mathbb{D}}}
\def\P{\mathrm{P}}

\newcounter{znum}
\newcommand{\z}[1]{\addtocounter{znum}{1} \textbf{Задача \arabic{znum}#1. }}

\newcounter{defnum}
\newcommand{\df}[1]{\addtocounter{defnum}{1} \textbf{Определение \arabic{defnum}.} {\it #1}}

\renewcommand{\!}{\textcolor{red}{!}}


\begin{document}
	
\thispagestyle{empty}
\begin{center}
	\large
	\textbf{Задачи по терверу}
	\normalsize
\end{center}


\begin{center}
	\textbf{Распределения}
\end{center}

\z{} Где находится $\alpha$-тая квантиль у равномерного распределения на [-1, 1]?

\z{} Нарисуйте ф-ции распределения и плотности (если она есть) для следующих случайных величин:  

\textbf{а)} Бернуллиевская  

\textbf{б)} Дискретная случайная величина, принимающая значение $a_i$ с вероятностью $p_i$  

\textbf{в)} Раномерное на [a, b]  

\textbf{г*)} Сумма двух независимых равномерных на [0, 1]  

\textbf{д*)} Максимум из двух независимых равномерных на [0, 1]  

\z{} Как сгенерировать на компе сэмпл из случайной величины с ф-цией распределения $F$, если известно как посчитать $F^{-1}$?

\z{} Есть два распределения, мы кидаем монетку и если выпал орел сэмплируем число из первого распределения, если решка -- из второго. Как выражается функция распределения полученной случайной величины, через функции распределения исходных величин?

\z{} Как изменится функция распределения случайной величины, если к ней добавить \textbf{а)} константу? \textbf{б)} бернуллиевскую случайную величину с вероятностью единицы равной p?

\z{*} Как связаны функции плотности и распределения?

\begin{center}
	\textbf{Про тест Манна-Уитни}
\end{center}

\z{} Убедитесь, что статистику Манна-Уитни можно посчитать так: объединить две выборки в одну большую выборку и отсортировать ее. Затем для каждого начального отрезка большой выборки нарисовать точку с координатами $(n_1, n_2)$, где $n_1$ и $n_2$ -- количества объектов в начальном отрезке, пришедших из первой и второй выборки соответственно. И наконец соединить построенные точки ломаной и посчитать площадь под ее графиком. Какова вычислительная сложность этого алгоритма при аккуратной реализации?

\z{} Приведите пример таких величин X, Y что $\P(Y > X) > 1/2$, но $\E(Y) < \E(X)$.

\z{} Приведите пример таких величин X, Y и Z, что 
$$\P(Y > X) > 1/2,\quad  \P(Z > Y) > 1/2 \quad \mbox{и} \quad \P(X > Z) > 1/2.$$

\z{*} Посчитайте дисперсию статистики Манна-Уитни при условии нулевой гипотезы.

\begin{center}
	\textbf{Разное.}
\end{center}


\z{*} Пусть $X_1, \ldots , X_n$ -- независимые одинаково распределенные величины с матожиданием $m$ и дисперсией $d$. Пусть $\hat m = \frac{1}{n} \sum_{i = 1}^n X_i$.

\textbf{а)} Посчитайте матожидание выборочной дисперсии:
$$ \hat d = \frac1n \sum_{i=1}^n (X_i - \hat m)^2$$

\textbf{б)} Предложите оценку дисперсии, такую что ее матожидание равно $d$.

\z{**} Посчитайте правильное pvalue для "теста Гальтона".

\end{document}