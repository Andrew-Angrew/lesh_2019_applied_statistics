
\documentclass[12pt,a4paper]{article}

\usepackage[T2A]{fontenc} % Поддержка русских букв
\usepackage[utf8]{inputenc} % Кодировка utf8
\usepackage[english, russian]{babel} % Языки: русский, английский
% \usepackage{pscyr} % Нормальные шрифты

\usepackage{amsmath, amssymb}
\usepackage{amscd}
\usepackage{amsthm}
\usepackage[left=1.5cm,right=2cm,top=1.5cm,bottom=1.5cm]{geometry}
\usepackage{geometry}
\usepackage{fancyhdr}
\usepackage{color}
\usepackage{graphicx}
\graphicspath{ {./pic/} }

\usepackage{xcolor}
\usepackage{hyperref}
% Цвета для гиперссылок
\definecolor{linkcolor}{HTML}{799B03} % цвет ссылок
\definecolor{urlcolor}{HTML}{799B03} % цвет гиперссылок
\hypersetup{pdfstartview=FitH,  linkcolor=linkcolor,urlcolor=urlcolor, colorlinks=true}

\def\Z{\mathbb{Z}}
\def\N{\mathbb{N}}
\def\R{\mathbb{R}}
\def\F{\mathbb{F}}
\def\Q{\mathbb{Q}}

\def\E{\mathrm{\mathbb{E}}}
\def\D{\mathrm{\mathbb{D}}}
\def\P{\mathrm{P}}

\newcounter{znum}
\addtocounter{znum}{11}
\newcommand{\z}[1]{\addtocounter{znum}{1} \textbf{A\arabic{znum}#1. }}

\newcounter{defnum}
\newcommand{\df}[1]{\addtocounter{defnum}{1} \textbf{Определение \arabic{defnum}.} {\it #1}}

\renewcommand{\!}{\textcolor{red}{!}}


\begin{document}
	
\thispagestyle{empty}
\begin{center}
	\large
	\textbf{Задачи на аналитику}
	\normalsize
\end{center}



Многие формулировки довольно размытые, уточняйте устно!

Задачи A1-A11 находятся в \href{https://github.com/Andrew-Angrew/lesh_2019_applied_statistics/blob/master/materials/Nauka_i_zhizn.pdf}{этой pdf-ке}.



\z{} В начале Первой мировой войны в униформу британских солдат входила коричневая матерчатая фуражка. Металлических касок у них не было. Через некоторое время командование армии было обеспокоено большим количеством ранений в голову. Было решено заменить фуражку металлической каской. Но вскоре командование было удивлено, узнав, что количество ранений в голову увеличилось. Необходимо заметить, что интенсивность сражений была примерно одинаковой до и после введения касок. Так почему же число ранений в голову увеличилось, когда солдаты стали надевать каски, а не фуражки?

\z{} Периодически вузы рассылают своим выпускникам анкеты с вопросами об успешности их карьеры. Почему рэйтинги успешности выпускников, основанные на этих анкетах получаются слишком "оптимистичными"?

\begin{center}
	\textbf{Более математичные задачи.}
\end{center}

\z{} Как вы оцените среднее количество пассажиров в самолетах, прилетающих в аэропорт, если вы можете только опрашивать прилетающих пассажиров о том, сколько пассажиров было в самолете, которым они прилетели?

\z{} Как рассчитать размер эксперимента, исходя из ожидаемой силы эффекта, который хочется доказать?

\z{} Все-таки хочется уметь досрочно отключать особо плохие эксперименты. Предложите честное решение этой проблемы (сколь угодно костыльное и странное)

\z{} Есть несколько независимых экспериментов, как сделать одно pvalue для гипотезы, что где-то эффект есть? (другими словами как настроить оповещение, которое разбудит тебя ночью, если у тебя в проде что-то сломалось)

\z{} Как должно быть распределено pvalue при условии нулевой гипотезы?

\z{} Обобщите рассказанный тест Вальда на случай независимых выборок разного размера.

\end{document} 